\documentclass{article}

\usepackage{array}

\title{Meerkats File Synchronizer - Final Report}
\author{Made in GB}
\date{}

\begin{document}
\maketitle{}

\section{Introduction}
Describe the context for the work and the problem you are addressing. Briefly summarise what you achieved in the project.

\section{Review}
Describe and cite related work.

\section{Requirements and design}
Describe the requirements you set for your project at the beginning and the design you have taken for your project. Focus on why you decided to tackle the problem in the way you did, and what effects that had on the design. You may also wish to mention the impact of team-working on your requirements and design.

\section{Implementation}
Describe the most significant implementation details, focussing on those where unusual or detailed solutions were required. Quote code fragments where necessary, but remember that the examiners have full access to your source code. Explain how you tested your software (e.g. unit testing) and the extent to which you tested it. If relevant to your project, explain performance issues and how you tackled them.

\section{Team work}
Describe how you worked together, including the tools and processes you used to facilitate group work.

\section{Evaluation}
Critically evaluate your project: what worked well, and what didn’t? how did you do relative to your plan? what changes were the result of improved thinking and what changes were forced upon you? how did your team work together? etc. Note that you need to show that you understand the weaknesses in your work as well as its strengths. You may wish to identify relevant future work that could be done on your project.

\section{Peer assessmen}
In a simple table, allocate the 100 ‘points’ you are given to each team member. Valid values range from 0 to 100 inclusive. You may assign decimal values, but the entire points must add up to precisely 100. An exception will be made if the 100 points are evenly divided between team members, where recurring decimals presented to 2 decimal places will be interpreted as if they added up to 100 (e.g. for 6 members it is acceptable for all students to be allocated precisely 16.66 points).


\begin{center}
\begin{tabular}{ | m{3cm}| m{1.3cm} | }
\hline
\textbf{Name} & \textbf{Points}  \\
\hline
Boyang Zhang & 16.66  \\
\hline
Xi He & 16.66  \\
\hline
YiFeng Zheng & 16.66 \\
\hline
Yenan Huang & 16.66 \\
\hline
Frida Solheim & 16.66 \\
\hline
Samah Alghamdi & 16.66 \\
\hline
\end{tabular}
\end{center}


\end{document}
